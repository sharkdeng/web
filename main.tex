\documentclass[12pt, a4paper]{report}
\usepackage[top=2cm, left=2cm, right=2cm, bottom=2cm]{geometry}
\usepackage{fontspec} % require LuaLatex engine
\setmainfont{Times New Roman}
\usepackage{enumerate}
\usepackage{float}
% code
\usepackage{listings} % insert code
\usepackage{xcolor}	 % code color
\definecolor{codegreen}{rgb}{0, 0.6, 0}
\definecolor{codegray}{rgb}{0.5,0.5,0.5}
\definecolor{codepurple}{rgb}{0.58,0,0.82}
\definecolor{backcolour}{rgb}{0.95,0.95,0.92}
\lstset{
backgroundcolor=\color{backcolour}, 
commentstyle=\color{codegreen},
keywordstyle=\color{magnenta},
numbers=left, %行号在左侧显示
numberstyle= \small\color{codegray},%行号字体
numbersep=5pt,
rulesepcolor= \color{gray!80}, %代码块边框颜色
breaklines=true,  %代码过长则换行
keywordstyle= \color{blue},%关键字颜色
frame=shadowbox,	%用方框框住代码块
showspaces=false,
showstringspaces=false,
showtabs=false,
tabsize=1
}
\usepackage{hyperref}





\title{\textbf{Website Build Handbook}}
\author{Limin Deng}
\date{May 2019}

\begin{document}
\maketitle

\tableofcontents{}
\clearpage


\section{Installation}
\begin{enumerate}
    \item \textbf{Mysql}: 
        \begin{enumerate}
            \item download mysql community version and install. It will have an icon in \textbf{System Preferences}
            \item add environment path: \colorbox{gray!30}{export PATH="/usr/local/mysql/bin:\$PATH"} in ~/.bash\_profile
        \end{enumerate}
    \item \textbf{Nodejs}: download from official website
    \item \textbf{Phpmyadmin}: 
        \begin{enumerate}
            \item download \underline{\href{https://www.phpmyadmin.net}{Phpmyadmin}}
            \item put the folder in nginx doc root (if php config above is correct, phpmyadmin page will show)
            \item link to mysql
                \begin{enumerate}
                    \item find mysql.socket: \colorbox{gray!30}{mysql\_config \-\-socket} or \colorbox{gray!30}{mysql -u root -p }, after entering, \colorbox{gray!30}{mysql>> status}
                    \item modify \colorbox{gray!30}{/private/etc/php.ini}
                    \begin{lstlisting}
pdo_mysql.default_socket = /tmp/mysql.sock
mysqli.default_socket = /tmp/mysql.sock
                    \end{lstlisting}
                    \item restart php-fpm
                    
                    \item copy \colorbox{gray!30}{phpmyadmin/ config.sample.ini.php} to \colorbox{gray!30}{config.ini.php}
                    
                    \item modify \colorbox{gray!30}{phpmyadmin/libraries/config.default.php}
                    \begin{lstlisting}
$cfg['Servers'][$i]['user'] = 'root';
$cfg['Servers'][$i]['password'] = 'your password';
                    \end{lstlisting}
                \end{enumerate}
        \end{enumerate}
    \item \textbf{Boot}: After setting up the above steps, whenever restart computer, you have to type \colorbox{gray!30}{nginx}(start server) and \colorbox{gray!30}{php-fpm}(start php support). \\
    One method is to set boot 
        \begin{enumerate}
            \item create a run.sh in ~/
            \item modify file access authority: \colorbox{gray!30}{sudo chmod 777 run.sh}
            \item modify the file is open with terminal (Get Info)
            \item System Preferences/Users\&Groups/Login Items/add the file and click \underline{hide}.
        \end{enumerate}
        
    \item \textbf{Apache}: Mac built-in

    
\end{enumerate}


\section{Languages}
\begin{enumerate}
    \item Frontend: 
    \href{}{\textcolor{blue}{HTML}}, \href{}{\textcolor{blue}{CSS}}, \href{}{\textcolor{blue}{Javascript}}
    
    \item Backend: 
    \href{}{\textcolor{blue}{Php}}, \href{https://www.kaggle.com/muerbingsha/n-python}{\textcolor{blue}{Python}}
\end{enumerate}


\section{Resources}
    \begin{enumerate}
        \item \textbf{Wordpress}: content management system
            \begin{enumerate}
                \item download the package and put it in doc root
                \item if there is \underline{wp\_config.php}, delete it and start installation(config database).
            \end{enumerate}
        \item \textbf{Discourse}: online forum
        \item \textbf{Moodle}: learning management system 
        \item \textbf{Disqus}: Comment System
    \end{enumerate}



\chapter{Server}
\section{Apache}
    \subsection{Config File Paths}
    For original Apache:
    \begin{enumerate}
        \item config file: \colorbox{gray!30}{/etc/apache2/httpd.conf}
    \end{enumerate}
    
    For \colorbox{gray!30}{brew install httpd}
    \begin{enumerate}
        \item 
    \end{enumerate}
    
    % port
    \subsection{Change Port}
    Change this part of config file. Port 81 and 82 work as same as port 8000.
\begin{lstlisting}
<IfDefine SERVER_APP_HAS_DEFAULT_PORTS>
    Listen 8080
</IfDefine>
<IfDefine !SERVER_APP_HAS_DEFAULT_PORTS>
    Listen 8000
    Listen 81
    Listen 82
</IfDefine>
\end{lstlisting}
    
    % hostname
    \subsection{Specify Hostname}
    This includes two steps. First is to modify Apache config file.
\begin{lstlisting}
# ServerName gives the name and port that the server uses to identify itself.
# This can often be determined automatically, but we recommend you specify
# it explicitly to prevent problems during startup.
#
# If your host doesn't have a registered DNS name, enter its IP address here.
#
# opened by Shark
ServerName www.assginment.com:8000
\end{lstlisting}
Second is add this snippet to \textcolor{blue}{/etc/hosts}:
\begin{lstlisting}
127.0.0.1       www.assignment.com
\end{lstlisting}

\textcolor{blue}{/etc/hosts} is a file (Description about /etc/hosts in Linux Environment 2017) responsible for swiftly analyzing ip and domain name. 
Priority from the highest to the lowest is DNS Cache $\rightarrow$ /etc/hosts $\rightarrow$ DNS Server.


    % docroot
    \subsection{Change Docroot}
index.html is put default docroot \textcolor{blue}{/Library/WebServer/Documents}. \\
There are 3 steps to change this.  \\
First, change \textcolor{blue}{DocumentRoot} 
\begin{lstlisting}
# DocumentRoot: The directory out of which you will serve your
# documents. By default, all requests are taken from this directory, but
# symbolic links and aliases may be used to point to other locations.
#
# DocumentRoot "/Library/WebServer/Documents"
# <Directory "/Library/WebServer/Documents">
# Modified by Shark
DocumentRoot "/Users/sj/Documents/zzz/wp"
<Directory "/Users/sj/Documents/zzz/wp">
    #
    # Possible values for the Options directive are "None", "All",
    # or any combination of:
    #   Indexes Includes FollowSymLinks SymLinksifOwnerMatch ExecCGI MultiViews
    #
    # Note that "MultiViews" must be named *explicitly* --- "Options All"
    # doesn't give it to you.
    #
    # The Options directive is both complicated and important.  Please see
    # http://httpd.apache.org/docs/2.4/mod/core.html#options
    # for more information.
    #
    # Options FollowSymLinks Multiviews
    # added by Shark
    Options FollowSymLinks Multiviews

    MultiviewsMatch Any

    #
    # AllowOverride controls what directives may be placed in .htaccess files.
    # It can be "All", "None", or any combination of the keywords:
    #   AllowOverride FileInfo AuthConfig Limit
    #
    AllowOverride None

    #
    # Controls who can get stuff from this server.
    #
    Require all granted
</Directory>
\end{lstlisting}

Setting are detailed as follows:
\begin{enumerate}
\item{\textbf{Options}
\begin{enumerate}
\item \textbf{All}: All server characteristics except for MutliViews.
\item \textbf{None}: Not start server characteristics.
\item \textbf{Indexes}: If there are no default files which are designated in DirectoryIndex, the server will return a directory list which is generated by mod\_autoindex modular.
\item \textbf{FollowSymLinks}: Allow server to use symbol connection.
\item \textbf{Multiviews}: Allow server to provide multiple 
\item \textbf{ExecCGI}: Allow mod\_cgi to execute CGI script.
\item \textbf{Includes}: Allow mod\_include.
\item \textbf{IncludesNOEXEC}
\item \textbf{SymLinksIfOwnerMatch}
\end{enumerate}
}
\item{\textbf{AllowOverride}}
\item{\textbf{Require all granted}}
\end{enumerate}

Secondly, change \textcolor{blue}{DirectoryIndex}
These are default access files
\begin{lstlisting}
#
# DirectoryIndex: sets the file that Apache will serve if a directory
# is requested.
#
<IfModule dir_module>
    DirectoryIndex index.html index.php
</IfModule>
\end{lstlisting}


Thirdly, change \textcolor{blue}{User} and \textcolor{blue}{Group}. Or 403 Permission Error will occur. \par
To see what the use and group of website directory, use \colorbox{gray!30}{ls -l} and we get: 
\begin{lstlisting}
-rwxrwxrwx@ 1 sj  staff  61 Apr  9 12:51 index.html
\end{lstlisting}
\begin{enumerate}
\item \textbf{sj} is User
\item \textbf{staff} is Group
\item \textbf{-rwxrwxrwx} is file permission. 
\begin{table}[H]
\begin{tabular}{|c|c|c|}
\hline
r & 4 & 100 \\
\hline
w & 2 & 010 \\
\hline
x & 1 & 001 \\
\hline
none & 0 & 000 \\
\hline
\multicolumn{3}{|c|}{rwxrwxrwx} \\
\hline
File owner & Other users in same group with file owner & Other users not in same group with file owner  \\
\hline
\end{tabular}
\end{table}
\end{enumerate}
Change file permission by \colorbox{gray!30}{chmod 777 <filename>} on terminal.
\begin{lstlisting}
<IfModule unixd_module>
#
# If you wish httpd to run as a different user or group, you must run
# httpd as root initially and it will switch.
#
# User/Group: The name (or #number) of the user/group to run httpd as.
# It is usually good practice to create a dedicated user and group for
# running httpd, as with most system services.
#
#User _www
#Group _www
# added by Shark
User sj
Group staff
</IfModule>
\end{lstlisting}

    % docroot
    \subsection{Change php}
    From built-in php to brew php
    \begin{enumerate}
        \item \colorbox{gray!30}{brew install php}
        \item \colorbox{gray!30}{brew search php}
        \item \colorbox{gray!30}{brew install php@7.2}
        \item See which version of php is installed by brew: \colorbox{gray!30}{brew list}
        \item How to change: \colorbox{gray!30}{brew info php@7.2}
        \item change extension dir in php.ini
    \end{enumerate}
\begin{lstlisting}
To enable PHP in Apache add the following to httpd.conf and restart Apache:
    LoadModule php7_module /usr/local/opt/php@7.2/lib/httpd/modules/libphp7.so

    <FilesMatch \.php$>
        SetHandler application/x-httpd-php
    </FilesMatch>

Finally, check DirectoryIndex includes index.php
    DirectoryIndex index.php index.html

The php.ini and php-fpm.ini file can be found in:
    /usr/local/etc/php/7.2/

php@7.2 is keg-only, which means it was not symlinked into /usr/local,
because this is an alternate version of another formula.

If you need to have php@7.2 first in your PATH run:
  echo 'export PATH="/usr/local/opt/php@7.2/bin:$PATH"' >> ~/.bash_profile
  echo 'export PATH="/usr/local/opt/php@7.2/sbin:$PATH"' >> ~/.bash_profile

For compilers to find php@7.2 you may need to set:
  export LDFLAGS="-L/usr/local/opt/php@7.2/lib"
  export CPPFLAGS="-I/usr/local/opt/php@7.2/include"


To have launchd start php@7.2 now and restart at login:
  brew services start php@7.2
Or, if you don't want/need a background service you can just run:
  php-fpm
\end{lstlisting}
    
    
    
    \subsection{Commands}
        \begin{enumerate}
            \item look version: \colorbox{gray!30}{httpd -v}
            \item \colorbox{gray!30}{apachectl -v} to check version
            \item start apache: \colorbox{gray!30}{sudo apachectl start} \\
            /System/Library/LaunchDaemons/org.apache.httpd.plist: service already loaded
            \item stop apache: \colorbox{gray!30}{sudo apachectl stop}
            \item restart apache: \colorbox{gray!30}{sudo apachectl restart}
            \item \colorbox{gray!30}{apachectl -S} to check which files are parsed
            \item \colorbox{gray!30}{apachectl -t -D DUMP\_MODULES} to see all loaded modules
            \item \colorbox{gray!30}{apachectl -t} to check httpd.conf syntax  
        \end{enumerate}
        
        
    \subsection{Dealing With Errors}
        \textbf{Cannot connect to the server} 
        \begin{enumerate}
        \item Check whether the port is taken by \colorbox{gray!30}{lsof -i:8000}
        \item Check whether network is ok by \colorbox{gray!30}{ping localhost}
        \item Check connection by \colorbox{gray!30}{curl -v http://localhost:8000}
        \item If above steps return ok, then try this http://www.assignment.com:8000/index.html on the browser, because sometimes http://www.assignment.com:8000 will not turn to that page.
        \end{enumerate}
        \textbf{403 Forbidden} 
        
        
    \subsection{Check}
        All instructions work fine on host (MacBook Pro) to get the screenshot (Figure \ref{fig-apache})
\begin{lstlisting}
curl -v HTTP://www.assignment.com:8000 (specified domain)
curl -v HTTP://www.127.0.0.1:8000  (lookback ip)
curl -v 10.20.129.180:8000 (local ip)
\end{lstlisting}
  
    

\newpage
\section{Nginx} 
Difference to Apache is that Apache is relatively slow while handling heavy load and processing large number of requests.
    \subsection{Installation}
        \colorbox{gray!30}{brew install nginx}
    \subsection{Config File Paths}
    \begin{enumerate}
        \item load all files in \colorbox{gray!30}{/usr/local/etc/nginx/servers/}
        \item config file with default port 8080: \colorbox{gray!30}{/usr/local/etc/nginx/nginx.conf} 
        \item doc root: \colorbox{gray!30}{/usr/local/var/www}
        \item \colorbox{gray!30}{/usr/local/Cellar/nginx/1.17.2}
    \end{enumerate}
    
    \subsection{Commands}
    \begin{enumerate}
        \item start nginx: \colorbox{gray!30}{brew services start nginx}
        \item start nginx: \colorbox{gray!30}{nginx} \\
        If 127.0.0.1:8080 can show nginx welcome page, then everything is fine.
        \item stop nginx: \colorbox{gray!30}{nginx -s stop}
        \item check config syntax: \colorbox{gray!30}{nginx -t}
        \item show configurations: \colorbox{gray!30}{nginx -V}
    \end{enumerate}
    
    \subsection{Config}
        \begin{lstlisting}
#user  nobody;
worker_processes  1;

#error_log  logs/error.log;
#error_log  logs/error.log  notice;
#error_log  logs/error.log  info;

#pid        logs/nginx.pid;


events {
    worker_connections  1024;
}


http {
    include       mime.types;
    default_type  application/octet-stream;

    #log_format  main  '$remote_addr - $remote_user [$time_local] "$request" '
    #                  '$status $body_bytes_sent "$http_referer" '
    #                  '"$http_user_agent" "$http_x_forwarded_for"';

    #access_log  logs/access.log  main;

    sendfile        on;
    #tcp_nopush     on;

    #keepalive_timeout  0;
    keepalive_timeout  65;

    #gzip  on;

    server {
        listen       8080;
        server_name  localhost;

        #charset koi8-r;

        #access_log  logs/host.access.log  main;

        location / {
	    root html;  # [shark] here to change doc root
        index  index.html index.htm index.php; # [shark] add index.php to support php, or error "forbidden" will show 
        }

        #error_page  404              /404.html;

        # redirect server error pages to the static page /50x.html
        #
        error_page   500 502 503 504  /50x.html;
        location = /50x.html {
            root   html;
        }

        # proxy the PHP scripts to Apache listening on 127.0.0.1:80
        #
        #location ~ \.php$ {
        #    proxy_pass   http://127.0.0.1;
        #}

        # pass the PHP scripts to FastCGI server listening on 127.0.0.1:9000
        # # [shark] open this block
        location ~ \.php$ {
            root           html; # [shark] php file root, or error "file not found" will show
            fastcgi_pass   127.0.0.1:9000; # [shark] php-fpm server location
            fastcgi_index  index.php;
        #    fastcgi_param  SCRIPT_FILENAME  /scripts$fastcgi_script_name;
            fastcgi_param   %document_root%fastcgi_script_name; # [shark] or error "file not found" will show
            include        fastcgi_params;
        }

        # deny access to .htaccess files, if Apache's document root
        # concurs with nginx's one
        #
        #location ~ /\.ht {
        #    deny  all;
        #}
    }


    # another virtual host using mix of IP-, name-, and port-based configuration
    #
    #server {
    #    listen       8000;
    #    listen       somename:8080;
    #    server_name  somename  alias  another.alias;

    #    location / {
    #        root   html;
    #        index  index.html index.htm;
    #    }
    #}


    # HTTPS server
    #
    #server {
    #    listen       443 ssl;
    #    server_name  localhost;

    #    ssl_certificate      cert.pem;
    #    ssl_certificate_key  cert.key;

    #    ssl_session_cache    shared:SSL:1m;
    #    ssl_session_timeout  5m;

    #    ssl_ciphers  HIGH:!aNULL:!MD5;
    #    ssl_prefer_server_ciphers  on;

    #    location / {
    #        root   html;
    #        index  index.html index.htm;
    #    }
    #}
    include servers/*;
}

        \end{lstlisting}
        
    \subsection{Problem}
        \begin{enumerate}
            \item \textbf{413 Request Entity Too Large} \\
                \begin{enumerate}[(1)]
                    \item add \colorbox{gray!30}{client\_max\_body\_size = 5m;} (default is 1m) in nginx.config
                    \item \colorbox{gray!30}{nginx -s reload}
                \end{enumerate}
            \item \textbf{502 Bad Gateway}
        \end{enumerate}
    


\newpage
\section{Nodejs}
\newpage
\section{Flask}
First, we need to install \textbf{Python} and \textbf{Flask}. \par
\begin{lstlisting}
conda install flask 
conda install flask-wtf 
conda install flask-script 
\end{lstlisting}
Secondly, we build file structures. \par
\begin{lstlisting}
root
--app
----static
----templates
--root.py
--v.flaskenv
\end{lstlisting}
Thirdly, to get an idea of flask language. \par 
\begin{enumerate}[(1)]
    \item \textbf{Variable:} \{\{ x \}\}
    \item \textbf{Loop:} \{\% for x in xs \%\} $\Longrightarrow$ \{\% endfor \%\}
    \item \textbf{If:} \{\% if x \%\} $\Longrightarrow$ \{\% else \%\} $\Longrightarrow$ \{\% endif \%\}
    \item \textbf{Block:} \{\% block content \%\} $\Longrightarrow$\{\% endblock \%\}
\end{enumerate}


\chapter{Database}
\section{Mysql}
    
    \subsection{Commands}
    \begin{enumerate}
        \item \colorbox{gray!30}{mysql --version}
        \item \colorbox{gray!30}{mysql -u root -p}
        \item Database
            \begin{enumerate}
                \item \colorbox{gray!30}{show databases;}
                \item \colorbox{gray!30}{create database <dbName>;}
                \item \colorbox{gray!30}{drop database <dbName>;}
                \item \colorbox{gray!30}{use <dbName>;}
            \end{enumerate}
        \item Table
            \begin{enumerate}
                \item \colorbox{gray!30}{show tables;}
                \item \colorbox{gray!30}{desc <tableName>;}
                \item \colorbox{gray!30}{drop table <tableName>;}
                \item \colorbox{gray!30}{alter table}
            \end{enumerate}
        \item Schema
            \begin{enumerate}
                \item \colorbox{gray!30}{create schema <name>'}
                \item \colorbox{gray!30}{show schemas;}
            \end{enumerate}
        \item \colorbox{gray!30}{quit;}
    \end{enumerate}
    
    \subsection{Engine}
    
    \subsection{Data Type}
    \begin{description}
        \item[Numeric]{
        	\begin{description}
        	\item
        	\item[TINYINT] 1 byte / Integer(-128 to 127, 0 to 255)
        	\item[SMALLINT] 2 bytes
        	\item[MEDIUMINT] 3 bytes
        	\item[INT] 4 bytes
        	\item[BIGINT] 8 bytes
        	\item[DECIMAL] fixed-point (M,X) / maximum(65, 30) / default(10, 0)
        	\item[FLOAT] floating-point
        	\item[DOUBLE] double-precision floating-point
        	\item[REAL]
        	\item[BIT]
        	\item[BOOLEAN]
        	\item[SERIAL]
        	\end{description}
        }
        \item[Date and time]{
        	\begin{description}
        	\item
        	\item[DATE] 3 bytes / YYYY-MM-DD
        	\item[TIME] 3 bytes / HH:MM:SS
        	\item[YEAR] 1 byte / YYYY
        	\item[DATETIME] 8 bytes / YYYY-MM-DD HH:MM:SS
        	\item[TIMESTAMP] 4 bytes / YYYYMMDD HHMMSS
        	\end{description}
        }
        \item[String]{
        	\begin{description}
        	\item
        	\item[CHAR]
        	\item[VARCHAR]
        	\item[TINYTEXT]
        	\item[TEXT]
        	\item[MEDIUMTEXT]
        	\item[LONGTEXT]
        	\item[BINAR]
        	\item[VARBINARY]
        	\item[TINYBLOB]
        	\item[MEDIUMBLOB]
        	\item[BLOB]
        	\item[LONGBLOB]
        	\item[ENUM]
        	\item[SET]
        	\end{description}
        }
        \item[Spatial]{
        	\begin{description}
        	\item[GEOMETRY]
        	\item[POINT]
        	\item[LINESTRING]
        	\item[POLYGON]
        	\item[MULTIPOINT]
        	\item[MULTILINESTRING]
        	\item[MULTIPOLYGON]
        	\item[GEOMETRYCOLLECTION]
        	\end{description}
        }
        \item[JSON]
    \end{description}

    \subsection{Engine}
    \begin{enumerate}
        \item Innodb
    \end{enumerate}
    
    \subsection{Problem Shoot}
    \begin{enumerate}
        \item 
    \end{enumerate}
    
    
    
    \subsection{Questions}
    
\newpage
\section{Postsql}
    \subsection{File Path}
        \begin{enumerate}
            \item \colorbox{gray!30}{/usr/local/var/postgres}
            \item \colorbox{gray!30}{/Library/PostgreSQL/11}
        \end{enumerate}


\chapter{Language}
\section{HTML}

\newpage
\section{CSS}
    \subsection{Sass}
    

\newpage
\section{Javascript}
    \subsection{Typescript}
    \subsection{JQuery}
    \subsection{Webpack}
\newpage
\section{PHP}
    \subsection{Installation}
        \subsubsection{Mac built-in}
        \begin{enumerate}[(1)]
            \item \colorbox{gray!30}{php -v} check php is installed.
            \item php-fpm
                \begin{enumerate}[(a)]
                    \item copy the three files \colorbox{gray!30}{/private/etc/php.ini}, \colorbox{gray!30}{/private/etc/php-fpm.conf}, \\ \colorbox{gray!30}{/private/etc/php-fpm.d/www.conf}
                    \item modify php-fpm.ini
                    \begin{lstlisting}
error_log = /usr/local/var/log/php-fpm.log
                    \end{lstlisting}
                    \item \colorbox{gray!30}{php-fpm} should work
                    \item if the ports are taken, try \colorbox{gray!30}{lsof -i:9000}, then \colorbox{gray!30}{kill -9 <pid>}
                \end{enumerate}
            \item nginx 
                \begin{enumerate}[(a)]
                    \item modify \colorbox{gray!30}{/usr/local/etc/nginx/nginx.conf}
                    \begin{lstlisting}
location / {
  root html;  # [shark] here to change doc root
  index  index.html index.htm index.php; # [shark] add index.php to support php, or error "forbidden" will show 
}
    ...
# pass the PHP scripts to FastCGI server listening on 127.0.0.1:9000
# [shark] open this block
location ~ \.php$ {
  root           html; # [shark] php file root, or error "file not found" will show
  fastcgi_pass   127.0.0.1:9000; # [shark] php-fpm server location
  fastcgi_index  index.php;
  fastcgi_param   %document_root%fastcgi_script_name; # [shark] or error "file not found" will show
  include        fastcgi_params;
}
                    \end{lstlisting}
                \end{enumerate}
        \end{enumerate}
        
        \subsubsection{Brew}
        
        
    \subsection{File Paths}
        \begin{enumerate}
            \item php-fpm: \colorbox{gray!30}{/private/etc/php-fpm.conf}
            \item Built-in
                \begin{enumerate}
                    \item command path: \colorbox{gray!30}{/usr/bin}
                    
                    \item extension: \colorbox{gray!30}{/usr/lib/php}
                    \item config: \colorbox{gray!30}{/private/etc/php.ini}
                    \item \colorbox{gray!30}{/usr/local/bin/php (which php)}
                    \item \colorbox{gray!30}{/usr/local/etc/php}
                    \item \colorbox{gray!30}{/usr/local/lib/php}
                \end{enumerate}
            \item Brew
                \begin{enumerate}
                    \item \colorbox{gray!30}{/usr/local/Cellar/php@7.2}
                \end{enumerate}
            \item Port
                \begin{enumerate}[(1)]
                    \item \colorbox{gray!30}{/opt/local/lib/php71/}
                \end{enumerate}
        \end{enumerate}
        
        
    \subsection{Commands}
        \begin{enumerate}
            \item look version: \colorbox{gray!30}{php -v}
            \item look modules: \colorbox{gray!30}{php -m}
            \item start php-fpm: \colorbox{gray!30}{php-fpm}
        \end{enumerate}
        
    
    \subsection{Config}
        \subsubsection{Modify PHP file upload limit}
            \begin{enumerate}
                \item use \colorbox{gray!30}{phpinfo()} in webdoc to find php.ini
                \item modify php.ini
\begin{lstlisting}
upload_max_filesize = 2M
max_file_uploads = 20            
\end{lstlisting}
                \item restart server 
                    \begin{enumerate}
                        \item apache: \colorbox{gray!30}{sudo apachectl restart}
                        \item nginx: \colorbox{gray!30}{nginx -s stop; nginx}
                    \end{enumerate}
                \item for nginx, need to restart php-fpm also
                    \begin{enumerate}
                        \item use ActivityMonitor find php-fpm pid
                        \item stop php-fpm \colorbox{gray!30}{kill [pid]}
                        \item start \colorbox{gray!30}{php-fpm}
                    \end{enumerate}
            \end{enumerate}
            
        \subsubsection{Support Mysql}
        
        
    \subsection{Extension(Pear/Pecl)}
        \begin{enumerate}
            \item install \colorbox{gray!30}{pear} or \colorbox{gray!30}{pecl} 
                \begin{lstlisting}
curl -O https://pear.php.net/go-pear.phar
sudo php -d detect_unicode=0 go-pear.phar
# change installation root directory to /usr/local/pear
                \end{lstlisting}
            \item verify installation
                \begin{lstlisting}
pear version
                \end{lstlisting}
            \item install enxtension
                \begin{lstlisting}
sudo pecl install intl
                \end{lstlisting}
            \item problem 1:
                \begin{lstlisting}
grep: /usr/include/php/main/php.h: No such file or directory
grep: /usr/include/php/Zend/zend_modules.h: No such file or directory
grep: /usr/include/php/Zend/zend_extensions.h: No such file or directory
Configuring for:
PHP Api Version:
Zend Module Api No:
Zend Extension Api No:
                \end{lstlisting}
                solutions
                \begin{lstlisting}
cd /Library/Developer/CommandLineTools/Packages/
open macOS_SDK_headers_for_macOS_10.14.pkg
                \end{lstlisting}
            \item problem 2:
                \begin{lstlisting}
PHP Api Version:         20160303
Zend Module Api No:      20160303
Zend Extension Api No:   320160303
Cannot find autoconf. Please check your autoconf installation and the
$PHP_AUTOCONF environment variable. Then, rerun this script.

ERROR: `phpize' failed
                \end{lstlisting}
                solution:
                \begin{lstlisting}
brew install autoconf
                \end{lstlisting}
            
            \item problem 3:
            make problem
        \end{enumerate}
        
    \subsection{Extension(Port)}

\chapter{Others}
\newpage
\section{Selenium}
\input{parts/seo.tex}
\input{parts/ui.tex}

\chapter{Frameworks}
\section{Wordpress}
    \subsection{Problem Shoot}
        \begin{enumerate}
            \item \textbf{forget admin password} \\
            Enter mysql dataset and modify it.
        \end{enumerate}
\newpage
\section{Moodle}
    \subsection{Installation Issues}
        \begin{enumerate}
            \item php intl extension 
            \item position of moodledata
        \end{enumerate}



\end{document}
