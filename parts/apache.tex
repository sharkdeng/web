\section{Apache}
    \subsection{Config File Paths}
    For original Apache:
    \begin{enumerate}
        \item config file: \colorbox{gray!30}{/etc/apache2/httpd.conf}
    \end{enumerate}
    
    For \colorbox{gray!30}{brew install httpd}
    \begin{enumerate}
        \item 
    \end{enumerate}
    
    % port
    \subsection{Change Port}
    Change this part of config file. Port 81 and 82 work as same as port 8000.
\begin{lstlisting}
<IfDefine SERVER_APP_HAS_DEFAULT_PORTS>
    Listen 8080
</IfDefine>
<IfDefine !SERVER_APP_HAS_DEFAULT_PORTS>
    Listen 8000
    Listen 81
    Listen 82
</IfDefine>
\end{lstlisting}
    
    % hostname
    \subsection{Specify Hostname}
    This includes two steps. First is to modify Apache config file.
\begin{lstlisting}
# ServerName gives the name and port that the server uses to identify itself.
# This can often be determined automatically, but we recommend you specify
# it explicitly to prevent problems during startup.
#
# If your host doesn't have a registered DNS name, enter its IP address here.
#
# opened by Shark
ServerName www.assginment.com:8000
\end{lstlisting}
Second is add this snippet to \textcolor{blue}{/etc/hosts}:
\begin{lstlisting}
127.0.0.1       www.assignment.com
\end{lstlisting}

\textcolor{blue}{/etc/hosts} is a file (Description about /etc/hosts in Linux Environment 2017) responsible for swiftly analyzing ip and domain name. 
Priority from the highest to the lowest is DNS Cache $\rightarrow$ /etc/hosts $\rightarrow$ DNS Server.


    % docroot
    \subsection{Change Docroot}
index.html is put default docroot \textcolor{blue}{/Library/WebServer/Documents}. \\
There are 3 steps to change this.  \\
First, change \textcolor{blue}{DocumentRoot} 
\begin{lstlisting}
# DocumentRoot: The directory out of which you will serve your
# documents. By default, all requests are taken from this directory, but
# symbolic links and aliases may be used to point to other locations.
#
# DocumentRoot "/Library/WebServer/Documents"
# <Directory "/Library/WebServer/Documents">
# Modified by Shark
DocumentRoot "/Users/sj/Documents/zzz/wp"
<Directory "/Users/sj/Documents/zzz/wp">
    #
    # Possible values for the Options directive are "None", "All",
    # or any combination of:
    #   Indexes Includes FollowSymLinks SymLinksifOwnerMatch ExecCGI MultiViews
    #
    # Note that "MultiViews" must be named *explicitly* --- "Options All"
    # doesn't give it to you.
    #
    # The Options directive is both complicated and important.  Please see
    # http://httpd.apache.org/docs/2.4/mod/core.html#options
    # for more information.
    #
    # Options FollowSymLinks Multiviews
    # added by Shark
    Options FollowSymLinks Multiviews

    MultiviewsMatch Any

    #
    # AllowOverride controls what directives may be placed in .htaccess files.
    # It can be "All", "None", or any combination of the keywords:
    #   AllowOverride FileInfo AuthConfig Limit
    #
    AllowOverride None

    #
    # Controls who can get stuff from this server.
    #
    Require all granted
</Directory>
\end{lstlisting}

Setting are detailed as follows:
\begin{enumerate}
\item{\textbf{Options}
\begin{enumerate}
\item \textbf{All}: All server characteristics except for MutliViews.
\item \textbf{None}: Not start server characteristics.
\item \textbf{Indexes}: If there are no default files which are designated in DirectoryIndex, the server will return a directory list which is generated by mod\_autoindex modular.
\item \textbf{FollowSymLinks}: Allow server to use symbol connection.
\item \textbf{Multiviews}: Allow server to provide multiple 
\item \textbf{ExecCGI}: Allow mod\_cgi to execute CGI script.
\item \textbf{Includes}: Allow mod\_include.
\item \textbf{IncludesNOEXEC}
\item \textbf{SymLinksIfOwnerMatch}
\end{enumerate}
}
\item{\textbf{AllowOverride}}
\item{\textbf{Require all granted}}
\end{enumerate}

Secondly, change \textcolor{blue}{DirectoryIndex}
These are default access files
\begin{lstlisting}
#
# DirectoryIndex: sets the file that Apache will serve if a directory
# is requested.
#
<IfModule dir_module>
    DirectoryIndex index.html index.php
</IfModule>
\end{lstlisting}


Thirdly, change \textcolor{blue}{User} and \textcolor{blue}{Group}. Or 403 Permission Error will occur. \par
To see what the use and group of website directory, use \colorbox{gray!30}{ls -l} and we get: 
\begin{lstlisting}
-rwxrwxrwx@ 1 sj  staff  61 Apr  9 12:51 index.html
\end{lstlisting}
\begin{enumerate}
\item \textbf{sj} is User
\item \textbf{staff} is Group
\item \textbf{-rwxrwxrwx} is file permission. 
\begin{table}[H]
\begin{tabular}{|c|c|c|}
\hline
r & 4 & 100 \\
\hline
w & 2 & 010 \\
\hline
x & 1 & 001 \\
\hline
none & 0 & 000 \\
\hline
\multicolumn{3}{|c|}{rwxrwxrwx} \\
\hline
File owner & Other users in same group with file owner & Other users not in same group with file owner  \\
\hline
\end{tabular}
\end{table}
\end{enumerate}
Change file permission by \colorbox{gray!30}{chmod 777 <filename>} on terminal.
\begin{lstlisting}
<IfModule unixd_module>
#
# If you wish httpd to run as a different user or group, you must run
# httpd as root initially and it will switch.
#
# User/Group: The name (or #number) of the user/group to run httpd as.
# It is usually good practice to create a dedicated user and group for
# running httpd, as with most system services.
#
#User _www
#Group _www
# added by Shark
User sj
Group staff
</IfModule>
\end{lstlisting}

    % docroot
    \subsection{Change php}
    From built-in php to brew php
    \begin{enumerate}
        \item \colorbox{gray!30}{brew install php}
        \item \colorbox{gray!30}{brew search php}
        \item \colorbox{gray!30}{brew install php@7.2}
        \item See which version of php is installed by brew: \colorbox{gray!30}{brew list}
        \item How to change: \colorbox{gray!30}{brew info php@7.2}
        \item change extension dir in php.ini
    \end{enumerate}
\begin{lstlisting}
To enable PHP in Apache add the following to httpd.conf and restart Apache:
    LoadModule php7_module /usr/local/opt/php@7.2/lib/httpd/modules/libphp7.so

    <FilesMatch \.php$>
        SetHandler application/x-httpd-php
    </FilesMatch>

Finally, check DirectoryIndex includes index.php
    DirectoryIndex index.php index.html

The php.ini and php-fpm.ini file can be found in:
    /usr/local/etc/php/7.2/

php@7.2 is keg-only, which means it was not symlinked into /usr/local,
because this is an alternate version of another formula.

If you need to have php@7.2 first in your PATH run:
  echo 'export PATH="/usr/local/opt/php@7.2/bin:$PATH"' >> ~/.bash_profile
  echo 'export PATH="/usr/local/opt/php@7.2/sbin:$PATH"' >> ~/.bash_profile

For compilers to find php@7.2 you may need to set:
  export LDFLAGS="-L/usr/local/opt/php@7.2/lib"
  export CPPFLAGS="-I/usr/local/opt/php@7.2/include"


To have launchd start php@7.2 now and restart at login:
  brew services start php@7.2
Or, if you don't want/need a background service you can just run:
  php-fpm
\end{lstlisting}
    
    
    
    \subsection{Commands}
        \begin{enumerate}
            \item look version: \colorbox{gray!30}{httpd -v}
            \item \colorbox{gray!30}{apachectl -v} to check version
            \item start apache: \colorbox{gray!30}{sudo apachectl start} \\
            /System/Library/LaunchDaemons/org.apache.httpd.plist: service already loaded
            \item stop apache: \colorbox{gray!30}{sudo apachectl stop}
            \item restart apache: \colorbox{gray!30}{sudo apachectl restart}
            \item \colorbox{gray!30}{apachectl -S} to check which files are parsed
            \item \colorbox{gray!30}{apachectl -t -D DUMP\_MODULES} to see all loaded modules
            \item \colorbox{gray!30}{apachectl -t} to check httpd.conf syntax  
        \end{enumerate}
        
        
    \subsection{Dealing With Errors}
        \textbf{Cannot connect to the server} 
        \begin{enumerate}
        \item Check whether the port is taken by \colorbox{gray!30}{lsof -i:8000}
        \item Check whether network is ok by \colorbox{gray!30}{ping localhost}
        \item Check connection by \colorbox{gray!30}{curl -v http://localhost:8000}
        \item If above steps return ok, then try this http://www.assignment.com:8000/index.html on the browser, because sometimes http://www.assignment.com:8000 will not turn to that page.
        \end{enumerate}
        \textbf{403 Forbidden} 
        
        
    \subsection{Check}
        All instructions work fine on host (MacBook Pro) to get the screenshot (Figure \ref{fig-apache})
\begin{lstlisting}
curl -v HTTP://www.assignment.com:8000 (specified domain)
curl -v HTTP://www.127.0.0.1:8000  (lookback ip)
curl -v 10.20.129.180:8000 (local ip)
\end{lstlisting}
  
    
