\newpage
\section{PHP}
    \subsection{Installation}
        \subsubsection{Mac built-in}
        \begin{enumerate}[(1)]
            \item \colorbox{gray!30}{php -v} check php is installed.
            \item php-fpm
                \begin{enumerate}[(a)]
                    \item copy the three files \colorbox{gray!30}{/private/etc/php.ini}, \colorbox{gray!30}{/private/etc/php-fpm.conf}, \\ \colorbox{gray!30}{/private/etc/php-fpm.d/www.conf}
                    \item modify php-fpm.ini
                    \begin{lstlisting}
error_log = /usr/local/var/log/php-fpm.log
                    \end{lstlisting}
                    \item \colorbox{gray!30}{php-fpm} should work
                    \item if the ports are taken, try \colorbox{gray!30}{lsof -i:9000}, then \colorbox{gray!30}{kill -9 <pid>}
                \end{enumerate}
            \item nginx 
                \begin{enumerate}[(a)]
                    \item modify \colorbox{gray!30}{/usr/local/etc/nginx/nginx.conf}
                    \begin{lstlisting}
location / {
  root html;  # [shark] here to change doc root
  index  index.html index.htm index.php; # [shark] add index.php to support php, or error "forbidden" will show 
}
    ...
# pass the PHP scripts to FastCGI server listening on 127.0.0.1:9000
# [shark] open this block
location ~ \.php$ {
  root           html; # [shark] php file root, or error "file not found" will show
  fastcgi_pass   127.0.0.1:9000; # [shark] php-fpm server location
  fastcgi_index  index.php;
  fastcgi_param   %document_root%fastcgi_script_name; # [shark] or error "file not found" will show
  include        fastcgi_params;
}
                    \end{lstlisting}
                \end{enumerate}
        \end{enumerate}
        
        \subsubsection{Brew}
        
        
    \subsection{File Paths}
        \begin{enumerate}
            \item php-fpm: \colorbox{gray!30}{/private/etc/php-fpm.conf}
            \item Built-in
                \begin{enumerate}
                    \item command path: \colorbox{gray!30}{/usr/bin}
                    
                    \item extension: \colorbox{gray!30}{/usr/lib/php}
                    \item config: \colorbox{gray!30}{/private/etc/php.ini}
                    \item \colorbox{gray!30}{/usr/local/bin/php (which php)}
                    \item \colorbox{gray!30}{/usr/local/etc/php}
                    \item \colorbox{gray!30}{/usr/local/lib/php}
                \end{enumerate}
            \item Brew
                \begin{enumerate}
                    \item \colorbox{gray!30}{/usr/local/Cellar/php@7.2}
                \end{enumerate}
            \item Port
                \begin{enumerate}[(1)]
                    \item \colorbox{gray!30}{/opt/local/lib/php71/}
                \end{enumerate}
        \end{enumerate}
        
        
    \subsection{Commands}
        \begin{enumerate}
            \item look version: \colorbox{gray!30}{php -v}
            \item look modules: \colorbox{gray!30}{php -m}
            \item start php-fpm: \colorbox{gray!30}{php-fpm}
        \end{enumerate}
        
    
    \subsection{Config}
        \subsubsection{Modify PHP file upload limit}
            \begin{enumerate}
                \item use \colorbox{gray!30}{phpinfo()} in webdoc to find php.ini
                \item modify php.ini
\begin{lstlisting}
upload_max_filesize = 2M
max_file_uploads = 20            
\end{lstlisting}
                \item restart server 
                    \begin{enumerate}
                        \item apache: \colorbox{gray!30}{sudo apachectl restart}
                        \item nginx: \colorbox{gray!30}{nginx -s stop; nginx}
                    \end{enumerate}
                \item for nginx, need to restart php-fpm also
                    \begin{enumerate}
                        \item use ActivityMonitor find php-fpm pid
                        \item stop php-fpm \colorbox{gray!30}{kill [pid]}
                        \item start \colorbox{gray!30}{php-fpm}
                    \end{enumerate}
            \end{enumerate}
            
        \subsubsection{Support Mysql}
        
        
    \subsection{Extension(Pear/Pecl)}
        \begin{enumerate}
            \item install \colorbox{gray!30}{pear} or \colorbox{gray!30}{pecl} 
                \begin{lstlisting}
curl -O https://pear.php.net/go-pear.phar
sudo php -d detect_unicode=0 go-pear.phar
# change installation root directory to /usr/local/pear
                \end{lstlisting}
            \item verify installation
                \begin{lstlisting}
pear version
                \end{lstlisting}
            \item install enxtension
                \begin{lstlisting}
sudo pecl install intl
                \end{lstlisting}
            \item problem 1:
                \begin{lstlisting}
grep: /usr/include/php/main/php.h: No such file or directory
grep: /usr/include/php/Zend/zend_modules.h: No such file or directory
grep: /usr/include/php/Zend/zend_extensions.h: No such file or directory
Configuring for:
PHP Api Version:
Zend Module Api No:
Zend Extension Api No:
                \end{lstlisting}
                solutions
                \begin{lstlisting}
cd /Library/Developer/CommandLineTools/Packages/
open macOS_SDK_headers_for_macOS_10.14.pkg
                \end{lstlisting}
            \item problem 2:
                \begin{lstlisting}
PHP Api Version:         20160303
Zend Module Api No:      20160303
Zend Extension Api No:   320160303
Cannot find autoconf. Please check your autoconf installation and the
$PHP_AUTOCONF environment variable. Then, rerun this script.

ERROR: `phpize' failed
                \end{lstlisting}
                solution:
                \begin{lstlisting}
brew install autoconf
                \end{lstlisting}
            
            \item problem 3:
            make problem
        \end{enumerate}
        
    \subsection{Extension(Port)}